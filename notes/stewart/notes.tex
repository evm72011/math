\documentclass{article}
\usepackage[utf8]{inputenc}
\usepackage[a4paper, left=2cm, right=2cm, top=3cm, bottom=3cm]{geometry} % Adjust margins here
\usepackage{amsmath} % This package provides \text{ }

\title{Calculus}
\author{James Stewart} 
\date{\today}

\begin{document}
\maketitle % Generates the title, author, and date at the beginning of the document

\section*{1.4 Exponential Functions}
\subsection*{The number \( e \)}
Base for exponential function with tangent line 1 at 0.
\section*{2 Limits and Derivatives}
\subsection*{2.2 The Limit of a function}
\subsubsection*{Limit} 
Let \(f\) be a function defined on some open interval that contains the number \(a\), except possibly at \(a\) itself. Then we say that
the \textbf{limit of \( f(x) \) as \(x\) approaches \(a\) is \(L\)}, and we write
\[
\lim_{x \to a} f(x) = L
\]
This means:

\[
\forall \epsilon > 0, \, \exists \delta > 0 \, \text{ such that } \, 0 < |x - a| < \delta \implies |f(x) - L| < \epsilon.
\]

\subsubsection*{Left-Hand Limit} 
\[
\lim_{x \to a^-} f(x) = L
\]
\[
\forall \epsilon > 0, \, \exists \delta > 0 \, \text{ such that, if} \, a - \delta < x < a \text{ then } |f(x) - L| < \epsilon.
\]

\subsubsection*{Right-Hand Limit} 
\[
\lim_{x \to a^+} f(x) = L
\]
\[
\forall \epsilon > 0, \, \exists \delta > 0 \, \text{ such that, if} \, a < x < a + \delta \text{ then } |f(x) - L| < \epsilon.
\]

\subsubsection*{Infinite Limit} 
Let \(f\) be a function defined on some open interval that contains the number \(a\), except possibly at \(a\) itself. Then 
\[
\lim_{x \to a} f(x) = \infty
\]
This means:

\[
\forall M > 0, \, \exists \delta > 0 \, \text{ such that } \, 0 < |x - a| < \delta \implies f(x) > M.
\]

\subsubsection*{Limit Laws}
\begin{flalign}
\lim_{x \to a} [f(x) + g(x)] &= \lim_{x \to a} f(x) + \lim_{x \to a} g(x) \\
\lim_{x \to a} [f(x) - g(x)] &= \lim_{x \to a} f(x) - \lim_{x \to a} g(x) \\
\lim_{x \to a} [cf(x)] &= c \lim_{x \to a} [f(x)] \\
\lim_{x \to a} [f(x) \cdot g(x)] &= \left(\lim_{x \to a} f(x)\right) \cdot \left(\lim_{x \to a} g(x)\right) \\
\lim_{x \to a} \frac{f(x)}{g(x)} &= \frac{\lim\limits_{x \to a} f(x)}{\lim\limits_{x \to a} g(x)}, \quad \text{provided } \lim_{x \to a} g(x) \neq 0\\
\lim_{x \to a} [f(x)]^n &= [\lim_{x \to a} f(x)]^n \\
\lim_{x \to a} c &= c \\
\lim_{x \to a} x^n &= a^n \\
\lim_{x \to a} \sqrt[n]{x} &= \sqrt[n]{a} \\
\lim_{x \to a} \sqrt[n]{f(x)} &= \sqrt[n]{\lim_{x \to a} f(x)} \\
\end{flalign}

\subsubsection*{Theorem}
\begin{align*}
\lim_{x \to a} f(x) &= L \Leftrightarrow \lim_{x \to a^-} f(x) = L = \lim_{x \to a^+} f(x) \\
\end{align*}
Here you can write the main content of your document. You can add sections, subsections, lists, figures, tables, and more. Here are some examples:

\subsubsection*{Theorem}
If \( f(x) \leq g(x) \) when \( x \) is near \( a \) (except possibly at \( a \)) and the limits of \( f \) and \( g \) exist as x approaches \( a \), then
\[
\lim_{x \to a} f(x) \leq \lim_{x \to a} g(x)
\]

\subsubsection*{The Squeeze Theorem}
If \( f(x) \leq g(x) \leq h(x)\) when \( x \) is near \( a \) (except possibly at \( a \)) and
\[
\lim_{x \to a} f(x) = \lim_{x \to a} h(x) = L
\]
then
\[
\lim_{x \to a} g(x) = L
\]

\subsection*{Continuity}
A function \(f\) is continuous at a number \(a\) if
\[
\lim_{x \to a} f(x) = f(a)
\]

A function \(f\) is continuous from the right at a number \(a\) if
\[
\lim_{x \to a^+} f(x) = f(a)
\]
and \(f\) is continuous from the left at \(a\) if
\[
\lim_{x \to a^-} f(x) = f(a)
\]

\subsubsection*{Theorem} 
If \(f\) and \(g\) are continuous at \(a\) and \(a\) is constant, then the following functions are also continuous at \(a\)

\[
f + g \hspace{2cm} f-g \hspace{2cm} cf \hspace{2cm} fg \hspace{2cm} \frac{f}{g}  \hspace{.5cm} \text{if} \;  g(a) \neq 0
\]

\subsubsection*{Theorem} 
If \(f\) is continuous at \(b\) and \( \lim\limits_{x \to a} g(x) = b\), then  \( \lim\limits_{x \to a} f(g(x)) = f(b)\).
In other words

\[
\lim_{x \to a} f(g(x)) = f(\lim\limits_{x \to a} g(x))
\]

\subsubsection*{Theorem} 
If \(g\) is continuous at \(a\) \(f\) is continuous at \(g(a)\), then the composite function \(f\circ g\) given by \((f\circ g)(x) = f(g(x))\) is continuous at \(a\).

\subsubsection*{The Intermediate Value Theorem} 
Suppose that \(f\) is continuous on the closed interval \( \left[a,b\right] \) and let \(N\) be any number between \(f(a)\) and \(f(b)\), where \(f(a) \neq f(b)\). Then there exist a number \(c\) in \((a,b)\) such that \(f(c)=N\)

\subsubsection*{Limit at infinity}
Let \(f\) is function defined on some interval  \( (a, \infty ) \). Then 
\[
\lim_{x \to \infty} f(x) = L
\]
means that for \( \forall \epsilon > 0 , \exists \text{ number } N \), such that 
for \( \forall x > N \) , \( |f(x) - L| < \epsilon \)

\subsection*{2.7 Derivatives and Rates of Change}
The tangent line to the curve \(y=f(x)\) at the point \(P(a,f(a))\) is the line through \(P\) with slope 
\[
m = \lim_{x \to a} \frac{f(x)-f(a)}{x-a}
\]
provided that this limit exists.
\end{document}
