\documentclass{article}
\usepackage[utf8]{inputenc}
\usepackage[a4paper, left=2cm, right=2cm, top=3cm, bottom=3cm]{geometry} % Adjust margins here
\usepackage{amsmath} % This package provides \text{ }

\title{Calculus}
\author{James Stewart} 
\date{\today}

\begin{document}
\maketitle % Generates the title, author, and date at the beginning of the document

\section*{1.4 Exponential Functions}
\subsection*{The number \( e \)}
Base for exponential function with tangent line 1 at 0.
\section*{2 Limits and Derivatives}
\subsection*{2.2 The Limit of a function}
\subsubsection*{Limit} 
Let \(f\) be a function defined on some open interval that contains the number \(a\), except possibly at \(a\) itself. Then we say that
the \textbf{limit of \( f(x) \) as \(x\) approaches \(a\) is \(L\)}, and we write
\[
\lim_{x \to a} f(x) = L
\]
This means:

\[
\forall \epsilon > 0, \, \exists \delta > 0 \, \text{ such that } \, 0 < |x - a| < \delta \implies |f(x) - L| < \epsilon.
\]

\subsubsection*{Left-Hand Limit} 
\[
\lim_{x \to a^-} f(x) = L
\]
\[
\forall \epsilon > 0, \, \exists \delta > 0 \, \text{ such that, if} \, a - \delta < x < a \text{ then } |f(x) - L| < \epsilon.
\]

\subsubsection*{Right-Hand Limit} 
\[
\lim_{x \to a^+} f(x) = L
\]
\[
\forall \epsilon > 0, \, \exists \delta > 0 \, \text{ such that, if} \, a < x < a + \delta \text{ then } |f(x) - L| < \epsilon.
\]

\subsubsection*{Infinite Limit} 
Let \(f\) be a function defined on some open interval that contains the number \(a\), except possibly at \(a\) itself. Then 
\[
\lim_{x \to a} f(x) = \infty
\]
This means:

\[
\forall M > 0, \, \exists \delta > 0 \, \text{ such that } \, 0 < |x - a| < \delta \implies f(x) > M.
\]

\subsubsection*{Limit Laws}
\begin{flalign}
\lim_{x \to a} [f(x) + g(x)] &= \lim_{x \to a} f(x) + \lim_{x \to a} g(x) \\
\lim_{x \to a} [f(x) - g(x)] &= \lim_{x \to a} f(x) - \lim_{x \to a} g(x) \\
\lim_{x \to a} [cf(x)] &= c \lim_{x \to a} [f(x)] \\
\lim_{x \to a} [f(x) \cdot g(x)] &= \left(\lim_{x \to a} f(x)\right) \cdot \left(\lim_{x \to a} g(x)\right) \\
\lim_{x \to a} \frac{f(x)}{g(x)} &= \frac{\lim\limits_{x \to a} f(x)}{\lim\limits_{x \to a} g(x)}, \quad \text{provided } \lim_{x \to a} g(x) \neq 0\\
\lim_{x \to a} [f(x)]^n &= [\lim_{x \to a} f(x)]^n \\
\lim_{x \to a} c &= c \\
\lim_{x \to a} x^n &= a^n \\
\lim_{x \to a} \sqrt[n]{x} &= \sqrt[n]{a} \\
\lim_{x \to a} \sqrt[n]{f(x)} &= \sqrt[n]{\lim_{x \to a} f(x)} \\
\end{flalign}

\subsubsection*{Theorem}
\begin{align*}
\lim_{x \to a} f(x) &= L \Leftrightarrow \lim_{x \to a^-} f(x) = \lim_{x \to a^+} f(x) = L \\
\end{align*}
Here you can write the main content of your document. You can add sections, subsections, lists, figures, tables, and more. Here are some examples:

\subsubsection*{Theorem}
If \( f(x) \leq g(x) \) when \( x \) is near \( a \) (except possibly at \( a \)) and the limits of \( f \) and \( g \) exist as x approaches \( a \), then
\[
\lim_{x \to a} f(x) \leq \lim_{x \to a} g(x)
\]

\subsubsection*{The Squeeze Theorem}
If \( f(x) \leq g(x) \leq h(x)\) when \( x \) is near \( a \) (except possibly at \( a \)) and
\[
\lim_{x \to a} f(x) = \lim_{x \to a} h(x) = L
\]
then
\[
\lim_{x \to a} g(x) = L
\]

\subsection*{Continuity}
A function \(f\) is continuous at a number \(a\) if
\[
\lim_{x \to a} f(x) = f(a)
\]

A function \(f\) is continuous from the right at a number \(a\) if
\[
\lim_{x \to a^+} f(x) = f(a)
\]
and \(f\) is continuous from the left at \(a\) if
\[
\lim_{x \to a^-} f(x) = f(a)
\]

\subsubsection*{Theorem} 
If \(f\) and \(g\) are continuous at \(a\) and \(a\) is constant, then the following functions are also continuous at \(a\)

\[
f + g \hspace{2cm} f-g \hspace{2cm} cf \hspace{2cm} fg \hspace{2cm} \frac{f}{g}  \hspace{.5cm} \text{if} \;  g(a) \neq 0
\]

\subsubsection*{Theorem} 
If \(f\) is continuous at \(b\) and \( \lim\limits_{x \to a} g(x) = b\), then  \( \lim\limits_{x \to a} f(g(x)) = f(b)\).
In other words

\[
\lim_{x \to a} f(g(x)) = f(\lim\limits_{x \to a} g(x))
\]

\subsubsection*{Theorem} 
If \(g\) is continuous at \(a\) \(f\) is continuous at \(g(a)\), then the composite function \(f\circ g\) given by \((f\circ g)(x) = f(g(x))\) is continuous at \(a\).

\subsubsection*{The Intermediate Value Theorem} 
Suppose that \(f\) is continuous on the closed interval \( \left[a,b\right] \) and let \(N\) be any number between \(f(a)\) and \(f(b)\), where \(f(a) \neq f(b)\). Then there exist a number \(c\) in \((a,b)\) such that \(f(c)=N\)

\subsubsection*{Limit at infinity}
Let \(f\) is function defined on some interval  \( (a, \infty ) \). Then 
\[
\lim_{x \to \infty} f(x) = L
\]
means that for \( \forall \epsilon > 0 , \exists \text{ number } N \), such that 
for \( \forall x > N \) , \( |f(x) - L| < \epsilon \)

\subsection*{2.7 Derivatives and Rates of Change}
The tangent line to the curve \(y=f(x)\) at the point \(P(a,f(a))\) is the line through \(P\) with slope 
\[
m = \lim_{x \to a} \frac{f(x)-f(a)}{x-a}
\]
provided that this limit exists.

\subsubsection*{Definition}
The \textbf{derivative of a function \(\boldsymbol{f}\) at a number \(\boldsymbol{a}\)}, denoted by \(f'(a)\)
\[
f'(a) = \lim_{h \to 0} \frac{f(a+h) - f(a)}{h}
\]
if this limit exists.

\subsubsection*{The tangent line equation}
The tangent line to \(y=f(x)\) at \( (a,f(a)) \) is the line through \( (a,f(a)) \) whose slope is equalt to \(f'(a)\)
\[
y - f(a) = f'(a)(x-a)
\]

\subsubsection*{Definition}
A function \(f\) is \textbf{differentiable at \(\boldsymbol{a}\)} if \(f'(a)\) exists.

\subsubsection*{Theorem}
If \(f\) is differentiable at \(a\), then \(f\) is continuous at \(a\)

\subsubsection*{Derivatives of Trigonometric Functions}
\begin{equation*}
\begin{aligned}[c]
\frac{d}{dx} (\sin x) = \cos x	\\
\frac{d}{dx} (\cos x) = -\sin x	\\
\frac{d}{dx} (\tan x) = \sec^2 x
\end{aligned}
\qquad\qquad\qquad
\begin{aligned}[c]
\frac{d}{dx} (\csc x) = \csc x \cot	x \\
\frac{d}{dx} (\sec x) = \sec x \tan	x \\
\frac{d}{dx} (\cot x) = -\csc^2 x
\end{aligned}
\end{equation*}

\subsubsection*{Two Special Trigonometric Limits}
\begin{equation*}
\begin{aligned}[c]
\lim_{\theta \to 0} \frac{\sin \theta}{\theta} = 1
\end{aligned}
\qquad\qquad\qquad
\begin{aligned}[c]
\lim_{\theta \to 0} \frac{\cos \theta - 1}{\theta} = 0
\end{aligned}
\end{equation*}

\subsubsection*{Twe Chain Rule}
If \(g\) is differentiable at \(x\) and \(f\) is differentiable at \(g(x)\), then the composite function \(F=f \cdot g\) defined by \(F = f(g(x)) \) is differentiable at \(x\) and \(F'\) is given by the product
\[
F'(x) = f'(g(x)) \cdot g'(x)
\]
In Leibniz notation, if \(y = f(u)\) and \(u = g(x)\)
\[
\frac{dy}{dx} = \frac{dy}{du} \frac{du}{dx}
\]

\subsubsection*{Implicit Differentiation}
If \(x^2 + y^2 = 25\), find \( \frac{dy}{dx} \)
\[
\frac{d}{dx}(x^2 + y^2)=\frac{d}{dx}(25)
\]
\[
2x + 2y y' = 0 \\
\]
\[
y' = -\frac{x}{y}
\]
\subsubsection*{Derivatives of Logarithmic Functions}
\[
\frac{d}{dx}(\log_b x)=\frac{1}{x \ln b}
\]
The proof via implicit differentiation

\subsubsection*{The Number \(e\) as a Limit}
Let \(f(x)=\ln x\), \(f'(x)=\frac{1}{x} \) and \(f'(1)=1\)
\[
f'(1) = \lim_{h \to 0} \frac{f(1+h)-f(1)}{h} = \lim_{x \to 0} \frac{f(1+x)-f(1)}{x}
\]
\[
= \lim_{x \to 0} \frac{\ln (1+x) -\ln 1}{x} = \lim_{x \to 0} \frac{1}{x} \ln (1+x) = \lim_{x \to 0} \ln (1+x)^\frac{1}{x}=1
\]
\[
e = e^1 = e^{\lim_{x \to 0} \ln (1+x)^ \frac{1}{x}} = \lim_{x \to 0} (1+x)^ \frac{1}{x}
\]

\subsubsection*{Derivatives of Inverse Trigonometric Functions}
\begin{equation*}
\begin{aligned}[c]
\frac{d}{dx}(\sin^{-1} x) = \frac{1}{\sqrt{1-x^2}} 	\\
\frac{d}{dx}(\cos^{-1} x) = -\frac{1}{\sqrt{1-x^2}}	\\
\frac{d}{dx}(\tan^{-1} x) = \frac{1}{1+x^2}
\end{aligned}
\qquad\qquad\qquad
\begin{aligned}[c]
\frac{d}{dx}(\csc^{-1} x) = -\frac{1}{x \sqrt{x^2-1}} 	\\
\frac{d}{dx}(\sec^{-1} x) = \frac{1}{x \sqrt{x^2-1}}		\\
\frac{d}{dx}(\cot^{-1} x) = -\frac{1}{1+x^2}
\end{aligned}
\end{equation*}

\subsubsection*{Linearization and Approximation}
The linear function whose graph is this tangent line, that is, \(L(x)=f(a) + f'(a)(x-a)\) is called the linearization of \(f\) at \(a\). The approximation \(f(x) \approx L(X)\)

\subsubsection*{Derivatives of Hyperbolic Functions}
\begin{equation*}
\begin{aligned}[c]
\frac{d}{dx} (\sinh x) = \cosh x	\\
\frac{d}{dx} (\cosh x) = -\sinh x	\\
\frac{d}{dx} (\tanh x) = sech^2 x
\end{aligned}
\qquad\qquad\qquad
\begin{aligned}[c]
\frac{d}{dx} (csch x) = -csch x \coth x	\\
\frac{d}{dx} (sech x) = -sech x \tanh x \\
\frac{d}{dx} (\coth x) = -csch^2 x
\end{aligned}
\end{equation*}

\subsubsection*{Derivatives of Inverse Hyperbolic Functions}
\begin{equation*}
\begin{aligned}[c]
\frac{d}{dx}(\sinh^{-1} x) = \frac{1}{\sqrt{1+x^2}} 	\\
\frac{d}{dx}(\cosh^{-1} x) = \frac{1}{\sqrt{x^2-1}}	\\
\frac{d}{dx}(\tanh^{-1} x) = \frac{1}{1-x^2}
\end{aligned}
\qquad\qquad\qquad
\begin{aligned}[c]
\frac{d}{dx}(csch^{-1} x) = -\frac{1}{|x| \sqrt{x^2+1}} 	\\
\frac{d}{dx}(sech^{-1} x) = -\frac{1}{x \sqrt{1-x^2}}		\\
\frac{d}{dx}(\coth^{-1} x) = \frac{1}{1-x^2}
\end{aligned}
\end{equation*}

\subsubsection*{The Extreme Value Theorem}
If \(f\) is continuous on a closed interval \([a,b]\), then \(f\) attains an absolute maximum value \(f(c)\) and an absolute minimum value \(f(d)\) at some numbers \(c\) and \(d\) in \([a,b]\).

\subsubsection*{Ferma's Theorem}
If \(f\) has a local maximum or minimum at \(c\), and if \(f'(c)\) exists, then \(f'(c)=0\).

\subsubsection*{Definition}
A critical number of a function \(f\) is a number \(c\) in the domain of \(f\) such that either \(f'(c)=0\) or \(f'(c)\) does not exist.
If \(f\) has a local maximum or minimum at \(c\), then \(c\) is a critical number of \(f\).

\subsubsection*{Rolle's Theorem}
Let \(f\) be a function that satisfies the following three hypotheses:
\begin{enumerate}
  \item \(f\) is continuous on the closed interval \([a,b]\).
  \item \(f\) is differentiable on the open interval \((a,b)\).
  \item \(f(a)=f(b)\).
\end{enumerate}
Then there is a number \(c\) in \((a,b)\) such that \(f'(c)=0\).

\subsubsection*{The Mean Value Theorem}
Let \(f\) be a function that satisfies the following three hypotheses:
\begin{enumerate}
  \item \(f\) is continuous on the closed interval \([a,b]\).
  \item \(f\) is differentiable on the open interval \((a,b)\).
\end{enumerate}
Then there is a number \(c\) in \((a,b)\) such that 
\[
f'(c) = \frac{f(b)-f(a)}{b-a}
\]

\subsubsection*{Definition}
If the graph of \(f\) lies above all of its tangents on an interval \(I\), then \(f\) is called \textbf{concave upward} on \(I\). If the graph of \(f\) lies below all of its tangents on \(I\), then \(f\) is called \textbf{concave downward} on \(I\).

\subsubsection*{l’Hospital’s Rule}
Suppose \(f\) and \(g\) are differentiable and \(g'(x) \neq 0 \) on an open interval \(I\) that contains \(I\) (except possibly at \(a\)). Suppose that
\[
\lim_{x \to a} f(x) = 0 \text{ and } \lim_{x \to a} g(x) = 0
\]
or that 
\[
\lim_{x \to a} f(x) = \pm \infty \text{ and } \lim_{x \to a} g(x) = \pm \infty
\]
(In other words, we have an indeterminate form of type \( \frac{0}{0} \) or \( \frac{\infty}{\infty} \)) Then
\[
\lim_{x \to a} \frac{f(x)}{g(x)} = \lim_{x \to a} \frac{f'(x)}{g'(x)}
\]
if the limit on the right side exists (or is \(\infty\) or \(-\infty\)).

\subsubsection*{Indeterminate Products (Type \(0 \cdot \infty\))}
\[
f \cdot g = \frac{f}{1/g} = \frac{g}{1/f}
\]
such technique can be sometimes applied to the cases \( \infty - \infty\), \( 0^0\), \( \infty ^ 0\), \(1 ^ \infty \) (pages 314-315)

\subsubsection*{Definition}
A function \(F\) is called an antiderivative of \(f\) on an interval \(I\) if \(F'(x)=f(x)\) for all \(x\) in \(I\).

\subsubsection*{Theorem}
If \(F\) is an antiderivative of \(f\) on an interval \(I\), then the most general antiderivative of \(f\) on \(I\) is \(F(x) + C\) where \(C\) is an arbitrary constant.

\subsection*{Integrals}
\subsubsection*{Definition}
The \textbf{area} \(A\) of the region \(S\) that lies under the graph of the continuous function \(f\) is the limit of the sum of the areas of approximating rectangles
\[
A = \lim_{n \to \infty} R_n = \lim_{n \to \infty} [f(x_1) \Delta x + f(x_2) \Delta x + ... + f(x_n) \Delta x]
\]

\subsubsection*{Definition of a Definite Integral}
If \(f\) is a function defined for \(a \leq x \leq b\), we divide the interval \([a,b]\) into \(n\) subintervals of equal width \(\Delta x = (b-a)/n\). We let \(x_0 (= a), x_1,x_2,...,x_n(=b)\) be the endpoints of these subintervals and we let \(x_0^*,x_1^*,x_2^*,...,x_n^*\) be any \textbf{sample points} in these subintervals, so \(x_i^*\) lies in the \textit{i}th subinterval \([x_{i-1},x_i]\). Then the \textbf{definite integral of \(f\) from \(a\) to \( b\)} is
\[
\int_a^b f(x) \, dx = \lim_{n \to \infty} \sum_{i=1}^n f(x_i^*) \Delta x
\]
provided that this limit exists and gives the same value for all possible choices of
sample points. If it does exist, we say that \(f\) is \textbf{integrable} on \([a,b]\).

\subsubsection*{The Fundamental Theorem of Calculus, Part 1}
If \(f\) is continuous on \([a,b]\) then the function \(g\) defined by
\[
g(x) = \int_a^x f(t) \, dt	\qquad a \leq x \leq b
\]
is continuous on \([a,b]\) and differentiable on \((a,b)\), and \(g'(x) = f(x)\)

\subsubsection*{The Fundamental Theorem of Calculus, Part 2}
If \(f\) is continuous on \([a,b]\) then
\[
\int_a^b f(x) \, dx	= F(b) - F(a)
\]
where \(F\) is any antiderivative of \(f\), that is, a function \(F\) such that \(F' = f\).

\subsubsection*{Net Change Theorem}
The integral of a rate of change is the net change:
\[
\int_a^b F'(x) \, dx	= F(b) - F(a)
\]

\subsubsection*{The Substitution Rule}
If \(u=g(x)\) is a differentiable function whose range is an interval \(I\) and \(f\) is continuous on \(I\), then
\[
\int f(g(x))g'(x)\, dx = \int f(u)\, du
\]

\subsubsection*{The Substitution Rule for Definite Integrals}
If \(g'\) is continuous on \([a,b]\) and \(f\) is continuous on the range of \(u=g(x)\), then
\[
\int_a^b f(g(x))g'(x)\, dx = \int_{g(a)}^{g(b)} f(u)\, du
\]

\subsubsection*{The Mean Value Theorem for Integrals}
If \(f\) is continuous on \([a,b]\), then there exists a number \(c\) in \([a,b]\) such that 
\[
f(c) = f_{avg} = \frac{1}{b-a} \int_a^b f(x)\, dx
\]
or
\[
\int_a^b f(x)\, dx = f(c) (b - a)
\]

\subsubsection*{Integration by Parts}
\[
\int u\, dv = uv - \int v\, du
\]

\subsubsection*{Trigonometric Substitution}
\begin{equation*}
\begin{aligned}[c]
\sqrt{a^2 - x^2} \\
\sqrt{a^2 + x^2} \\
\sqrt{x^2 - a^2}
\end{aligned}
\qquad
\begin{aligned}[c]
x = a \sin \theta 	\\
x = a \tan \theta 	\\
x = a \sec \theta 	
\end{aligned}
\qquad
\begin{aligned}[c]
1 - \sin ^2 \theta = \cos ^2 \theta \\
1 + \tan ^2 \theta = \sec ^2 \theta \\
\sec ^2 \theta - 1 = \tan ^2 \theta
\end{aligned}
\end{equation*}

\subsection*{Differential Equations}
\end{document}
